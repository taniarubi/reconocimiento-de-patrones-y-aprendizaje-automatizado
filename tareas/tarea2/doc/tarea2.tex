\documentclass[letterpaper,11pt]{article}

% Soporte para los acentos.
\usepackage[utf8]{inputenc}
\usepackage[T1]{fontenc}    
% Idioma español.
\usepackage[spanish,mexico, es-tabla]{babel}
% Soporte de símbolos adicionales (matemáticas)
\usepackage{multirow}
\usepackage{amsmath}		
\usepackage{amssymb}		
\usepackage{amsthm}
\usepackage{amsfonts}
\usepackage{latexsym}
\usepackage{enumerate}
\usepackage{ragged2e}
\usepackage{graphicx}
\usepackage{hyperref}
% Modificamos los márgenes del documento.
\usepackage[lmargin=2cm,rmargin=2cm,top=2cm,bottom=2cm]{geometry}

\title{Facultad de Ciencias, UNAM \\ 
       Reconocimiento de patrones y aprendizaje automatizado \\ 
       Tarea 2}
\author{Rubí Rojas Tania Michelle}
\date{\today}

\begin{document}
\maketitle

\begin{enumerate}
    % Ejercicio 1.
    \item Cada una de las líneas en el documento \texttt{tarea2\_docs.txt} lo 
    vamos a considerar como un documento. Realiza la limpieza y preprocesamiento 
    necesarios para contestar las siguientes preguntas:
    \begin{itemize}
        % Ejercicio 1.a
        \item ¿En qué tópicos se pueden dividir?

        % Ejercicio 1.b
        \item ¿Qué documentos habla de México y su cultura?
    \end{itemize}

    % Ejercicio 2.
    \item Descarga el dataset de datos de sonar de la siguiente liga:
    \url{https://archive.ics.uci.edu/ml/machine-learning-databases/undocumented/
    connectionist-bench/sonar/sonar.all-data}

    Mismo que deberás explorar y entender lo que hacen sus atributos. 
    
    Lleva a cabo una clasificación utilizándo la regresión logística de los 
    datos. Luego, efectúa una reducción de dimensión y haz tu clasificación 
    otra vez. La comparación entre las clasificaciones la harás con las 
    métricas que tú elijas. 

    ¿Qué observas y por qué sucede esto?

    % Ejercicio 3.
    \item Diseña un experimento para determinar para qué umbral es posible tener 
    un conjunto de datos no balanceado.

    % Ejercicio 4.
    \item Reproduce la gráfica de la información mutua del seno contra él mismo 
    considerándo un corrimiento de $50$ posiciones.

    % Ejercicio 5.
    \item El dataset de \texttt{seatbelt.csv} representa una serie de datos 
    temporales de diversos atributos que hacen referencia a la mortalidad 
    asociada a los accidentes de tránsito en Gran Bretaña en el periodo de 
    $1969$ y $1984$. La legislación para utilizar el cinturón de seguridad de 
    manera obligatoria fue introducida el $31$ de enero de $1983$. Ajusta un 
    modelo lineal generalizado para determinar la probabilidad de morir en 
    un accidente de tránsito y responde las siguientes preguntas:
    \begin{itemize}
        % Ejercicio 5.a
        \item ¿Volver obligatorio el cinturón de seguridad disminuyó la 
        probabilidad de morir en algún accidente de tránsito?

        % Ejercicio 5.b
        \item ¿Qué otra conclusión puedes generar a partir del modelo que 
        ajustaste? 
    \end{itemize}

    Se espera que lleves a cabo la normalización o estandarización del 
    dataset, la transformación de las variables que consideres pertinente y 
    que propongas qué variables tienen más o menos importancia en la 
    probabilidad de morir en un accidente de tránsito.
\end{enumerate}

\end{document}
